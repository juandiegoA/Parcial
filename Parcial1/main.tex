\documentclass{article}
\usepackage[utf8]{inputenc}
\usepackage[spanish]{babel}
\usepackage{listings}
\usepackage{graphicx}
\graphicspath{ {images/} }
\usepackage{cite}

\begin{document}

\begin{titlepage}
    \begin{center}
        \vspace*{1cm}
            
        \Huge
        \textbf{Parcial 1}
            
        \vspace{0.5cm}
        \LARGE
        Subtítulo
            
        \vspace{1.5cm}
            
        \textbf{Juan Diego Arias Toro}
            
        \vfill
            
        \vspace{0.8cm}
            
        \Large
        Despartamento de Ingeniería Electrónica y Telecomunicaciones\\
        Universidad de Antioquia\\
        Medellín\\
        Abril de 2021
            
    \end{center}
\end{titlepage}

\tableofcontents
\newpage
\section{Análisis del problema}\label{intro}
En esencia el problema necesita la implementación de un circuito y el desarrollo de un código para arduino, por este motivo considero que lo más apropiado es tomar dos el proyecto en dos partes la primera es la construcción del circuito, comenzando por la matriz de 64 leds y posteriormente ir manejando la coneccion de integrados y de los pines del arduino teniendo en cuenta que para este paso es crucial tener en cuenta el desarrollo del código.

En el desarrollo del código no tengo una idea clara de como iniciar pero y est problema que voy a solucionar mientras avanzo. 


\section{Esquema de desarrollo} \label{contenido}
*Los primeros días se dedican a pensar cómo abordar el problema.

*Se comienza diseñar el circuito con la matriz de leds de 8x8.

*Se integran las conexiones a los integrados al circuito.

*Se adelanta más en el informe escrito en Overleaf.

*El tiempo restante se dedicará a la construcción y manipulación del código para el correcto funcionamiento del circuito.

*Finalmente se terminará el informe y se realizará el video solicitado y todo será adjuntado al repositorio y finalmente será enviada.

\section{Algoritmo implementado} \label{contenido}
\begin{lstlisting}
const int data = 2;
const int latch = 3;
const int clock = 4;

bool menu=false;//solo tiene dos estados
int arr[8];
char patron[200];
int **matriz;

int row[8] = {127, 191, 223, 239, 247, 251, 253, 254};
// columnH para limpiar todo
int columnH[8] = {0,0,0,0,0,0,0,0};

void setup()
{
  Serial.begin(9600);
  // 74HC595
  pinMode(data, OUTPUT); // data
  pinMode(latch, OUTPUT); // store
  pinMode(clock, OUTPUT); // shift
  
  pinMode(13, OUTPUT);
  Serial.println("Introduzca 'V' para verificacion, I para imagen y P para patrones.");
}



void verificacion(){
  for(int k = 0; k<100; k++)
  {
    for(int i=0; i<8; i++)
    {
      digitalWrite(latch, LOW);
      shiftOut(data, clock, LSBFIRST, 255);
      shiftOut(data, clock, LSBFIRST, row[i]);
      digitalWrite(latch, HIGH);
      delay(2);
    }
  }
}
void limpiar(){
  for(int k = 0; k<1; k++)
  {
    for(int i=0; i<8; i++)
    {
      digitalWrite(latch, LOW);
      shiftOut(data, clock, LSBFIRST, 0);
      shiftOut(data, clock, LSBFIRST, row[i]);
      digitalWrite(latch, HIGH);
      delay(2);
    }
  }
}
void Dibujar(){
   for(int k = 0; k<100; k++)
  {
    for(int i=0; i<8; i++)
    {
      digitalWrite(latch, LOW);
      shiftOut(data, clock, LSBFIRST, arr[i]);
      shiftOut(data, clock, LSBFIRST, row[i]);
      digitalWrite(latch, HIGH);
      delay(2);
    }
  }
}
void readBinaryString(char *s) {
  int result = 0;
  for(int j=0;j<8;j++){
    result=0;
  while(*s != ',' && *s!='\0') {
    result=result<<1; //desplaza los bits una posicion a la izquierda
    //por ejemplo 11100 queda 111000
    if(*s == '1'){
      result = result | 1;//convierte en 1 la posicion indicada
    }
    s++; //incrementa la direccion
  }
  s++;//avanza la direccion del apuntador para superar la coma y continuar
  arr[j]=result;//guarda el numero convertido
  }
}
void publicar(){

  bool salir=true;
  int NumeroPatrones=0;
  while(salir){
    if (Serial.available() > 0){
      NumeroPatrones = Serial.parseInt();
      Serial.println(NumeroPatrones);
      salir=false;
    }
  }
  salir=true;
  Serial.println("Ingrese tiempo");
  int Tiempo=0;
  while(salir){
    if (Serial.available() > 0){
      Tiempo = Serial.parseInt();
      salir=false;
    }
  }
  //creacion de la matriz dinamica de patrones a graficar
  matriz=new int*[8];
  for(int i=0;i<8;i++){
  matriz[i]=new int[NumeroPatrones];
  }
  int contadorPatrones=0;
  salir=true;
  Serial.println("Introduzca patron");
  while(salir){
    if (Serial.available() > 0){
      size_t count = Serial.readBytesUntil('\n', patron, 200);
 	  readBinaryString(patron);
      	for(int r=0;r<8;r++){
  		*(*(matriz+r)+contadorPatrones)=arr[r];//matriz [i][j]
  		}
      delay(1000);
      contadorPatrones++;
      Serial.println("Introduzca patron");
    }
    if(contadorPatrones>=NumeroPatrones){
      salir=false;
    }
  }
  //DIBUJANDO
   for(int p = 0; p<NumeroPatrones; p++)
  {
     int aux[8];
    for(int i=0; i<8; i++)
    {
      aux[i]=*(*(matriz+i)+p);
    }
     for(int d=0; d<120; d++){
       for(int d2=0; d2<8; d2++){
        digitalWrite(latch, LOW);
        shiftOut(data, clock, LSBFIRST, aux[d2]);
        shiftOut(data, clock, LSBFIRST, row[d2]);
        digitalWrite(latch, HIGH);
        delay(1);
     }
     }
     delay(Tiempo);
   }
}
void loop()
{
   if (Serial.available() > 0) {
    
     if(menu){
      size_t count = Serial.readBytesUntil('\n', patron, 200);
 
       delay(1000);
       //Serial.println(patron);
       readBinaryString(patron);//"11111111,11111111,11111111");
       menu=false;
       Dibujar();
       Serial.println("Introduzca 'V' para verificacion, I para imagen y P para patrones.");
     }
     else{

    int inByte = Serial.read();

    switch (inByte) {

      case 'V':
	   Serial.println("Verificando: ");
       verificacion();
       Serial.println("Verificado.");
       Serial.println("Introduzca 'V' para verificacion, I para imagen y P para patrones.");
		break;

      case 'I':
		Serial.println("Introduzca patron:");
        menu=true; //permite entrada menu
        break;
      case 'P':
		Serial.println("Introduzca numero de patrones:");
        publicar();
        Serial.println("Introduzca 'V' para verificacion, I para imagen y P para patrones.");
        break;

      default:

        // turn all the LEDs off:
      Serial.println("Introduzca una opcion valida");
      delay(1000);

    }
   }

  }
  limpiar();
  
}
\end{lstlisting}

\section{Problemas de desarrollo}\label{intro}
La toma y repartición de datos a través del serial fue un punto clave pero complicado en el desarrollo del parcial, pues no siempre se tenía claro cómo hacer el reparto de los tados hacia las funciones que podrían necesitarlo, esto debido al tamaño del código. Se obtuvo una solución satisfactoria por medio de una al implementar un arreglo que llevara los datos a la función utilizada.

La comunicación al arduino también forma parte de los grandes problemas que se tuvieron que afrontar en el parcial, pues el lenguaje del arduino se debía poder convertir los patrones binarios a patrones enteros para controlar las salidas del arduino.

La implementación de la función dinámica fue la solución al no saber como cuantos patrones pude ingresar el usuario por lo tanto no se podía reservar memoria desde antes del ingreso de los datos.

\section{Evolución del algoritmo}\label{intro}
Poder utilizar la matriz para que reciba los patrones, y de esta manera poder iluminar los patrones, eso se implementa por medio de dos arreglos uno para determinar columnas y filas. cuando la matriz ya recibe datos (iluminaba leds), se pasó a implementar la verificación para ello simplemente se coloca de salida 255 en cada línea. 

Después se desarrolló una función que leyera patrones, para esto se llegó a la solución de que se podría realizar con arreglos que recorrieran los patrones binarios hasta que se encuentre una coma.

después se desarrolló una matriz (de código) que almacena patrones para poder después realizar una presentación continua de patrones con un tiempo delimitado por el usuario, igualmente que los patrones aceptados por el arduino.

Poder tener una interacción con el usuario también fue un punto a tener en cuenta para pues la entrada de los textos y datos se debió plantear para dar una solución apropiada.

\section{Consideraciones a tener en cuenta en la implementación.
}\label{intro}
*El algoritmo no acepta numero negativo ni valores float.


*Ser cuidadoso con el arreglo de números binarios que se va a ingresar.


*Se permite ingresar al menos un numero binario pero de este debe ser de ocho dígitos, no más no menos.


*Al momento de ingresar el números de patrones deseados para la presentación deben ser un número entero (no se permiten char ni floats).
\end{document}
